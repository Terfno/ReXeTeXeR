% UTF-8
\documentclass[a4paper, xelatex]{bxjsarticle}
\XeTeXlinebreaklocale "ja_JP"
\XeTeXlinebreakskip=0em plus 0.1em minus 0.01em
\XeTeXlinebreakpenalty=0
\usepackage{xltxtra}

% fonts
\usepackage{fontspec}
\setmainfont[Scale=MatchLowercase]{NotoSerifJP-Regular.otf}
\setsansfont[Scale=MatchLowercase]{NotoSansJP-Regular.otf}
\setmonofont[Scale=MatchLowercase]{RobotoMono.ttf}

\usepackage{cite} % bibtex
\usepackage{graphicx} % 画像
\usepackage{here} % 図の強制配置
\usepackage{listings, jlisting} % ソースコード表示
\usepackage[normalem]{ulem} % 下線表示
\useunder{\uline}{\ul}{} % texデフォルト下線をulemの下線に置き換え
\usepackage{c_thesis} % 津山高専指定のスタイルファイル

% ソースコードの設定/listings の表示設定
\lstset{
  breaklines = true,
  tabsize = 2,
  frame=trbl, % 枠を上下左右に表示する
  showstringspaces=false, % 文字列中のスペースを" "と表示。def:可視
  numbers=left,
  framexleftmargin=6mm, % 行番号をフレーム内に
  numberstyle=\scriptsize, %行番号のサイズ
  stepnumber=1, % 1行おきに行番号を
  numbersep=1zw, % ソースと行番号の間隔
  language = C % 言語設定
}

% タイトル・表紙設定
\title{「研究タイトルっぽいやつ」}
\author{Takahito Sueda}
\date{2020年12月1日}
\kind{\interim} % 中間報告書の場合 / これはc_thesis.sty
% \kind{\preliminary} % 予備審査論文の場合 / これはc_thesis.sty
% \kind{\thesis} % 最終報告書の場合 / これはc_thesis.sty
\affiliation{情報システム系} % これはc_thesis.sty
\adviser{房 冠深} % これはc_thesis.sty

\begin{document}
  % 表紙
  \maketitle

  \begin{abstract}
    Abstractを書く。英語で。

    Lorem ipsum dolor sit amet, consectetur adipiscing elit, sed do eiusmod tempor incididunt ut labore et dolore magna aliqua. Ut enim ad minim veniam, quis nostrud exercitation ullamco laboris nisi ut aliquip ex ea commodo consequat. Duis aute irure dolor in reprehenderit in voluptate velit esse cillum dolore eu fugiat nulla pariatur. Excepteur sint occaecat cupidatat non proident, sunt in culpa qui officia deserunt mollit anim id est laborum.

    この論文の要約を英語のみを用いて、200ワード以上1ページ以内にまとめる。

    これすごくない?
  \end{abstract}

  % 目次
  \tableofcontents

  \section{はじめに}
  研究の必要性と意義、従来その分野の研究状況、研究の目的とその範囲、オリジナリティを主張する範囲などを記載する。
  現状についての共有→課題→解決方法と目標

  \section{先行研究}
  すでに存在している研究について記載する。

  \section{本論}

  \subsection{本論1}
  理論や実験方法、結果の解釈と考察などを明瞭に述べる。

  数式テスト
  \begin{eqnarray}
    2x_1 + x_2 & = & 5 \\
    2x_2 & = & 2
  \end{eqnarray}

  \subsection{本論2}
  文意や内容によって、節に分けて記述する。

  bibtexでの引用テスト。DL\cite{lecun2015deep}はML\cite{michie1994machine}に内包される1つの手法。

  \subsection{画像テスト}
  graphicx使う。
  \begin{center}
    \includegraphics[width=10cm]{img/logo.png} \\
    ReXeTeXeRのぶちかっこいいロゴ
  \end{center}

  \section{おわりに}
  研究目標に対する到達レベル、論文主張点のまとめ、今後の課題などを記述する。

  % 参考文献 / bibtex
  \bibliography{ref.bib}
  \bibliographystyle{junsrt}

  \appendix
  \section{付録の例}
  長い数式の誘導過程、実験装置やシステムについての詳しい説明など、本文中に挿入すると論旨が不明瞭になる事項は付録とする。
\end{document}
