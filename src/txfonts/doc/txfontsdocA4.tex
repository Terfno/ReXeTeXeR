\documentclass[11pt]{article}

\usepackage{a4wide,amsmath}
\usepackage{txfonts}
%\normalfont
%\usepackage[T1]{fontenc}
%\usepackage{textcomp}
\let\orgnonumber=\nonumber\usepackage{mathenv}\let\nonumb=\nonumber\let\nonumber=\orgnonumber

\allowdisplaybreaks

\newcommand{\bs}{\symbol{'134}}


\def\Ent#1{\csname #1\endcsname & \texttt{\bs #1}}
\def\EEnt#1#2{\csname #1\endcsname & \texttt{\bs #1},\,\texttt{\bs #2}}

\makeatletter
\newcount\curchar \newcount\currow \newcount\curcol
\newdimen\indexwd \newdimen\tempcellwd
\setbox0\hbox{\ttfamily0\kern.2em}
\indexwd=\wd0

\def\ident#1{#1}
\def\hexnumber#1{\ifcase\expandafter\ident\expandafter{\number#1} 0\or
1\or 2\or 3\or 4\or 5\or 6\or 7\or 8\or 9\or A\or B\or C\or D\or E\or
F\else ?\fi}

\def\rownumber{\ttfamily\hexnumber\currow}
\def\colnumber{\ttfamily\hexnumber\curcol \global\advance\curcol 1 }

\def\charnumber{\setbox0=\hbox{\char\curchar}%
  \ifdim\ht0>7.5pt\reposition
  \else\ifdim\dp0>2.5pt\reposition\fi\fi
  \box0 \global\advance\curchar1 }
\def\reposition{\setbox0=\hbox{$\vcenter{\kern1.5pt\box0\kern1.5pt}$}}

\def\dochart#1{%
  \begingroup
  \global\curchar=0 \global\currow=0 \global\curcol=0
  \def\hline{\kern2pt\hrule\kern3pt }%
  \setbox0\vbox{#1%
    \def\0{\hbox to\cellwd{\curcol}{\hss\charnumber\hss}}%
    \colnumbers
    \hline
    \setrow\setrow\setrow\setrow
    \hline
    \setrow\setrow\setrow\setrow
    \hline
    \colnumbers
  }%
  \vbox{%
    \hbox to\hsize{\kern\indexwd
      \def\fullrule{\hfil\vrule height\ht0 depth\dp0\hfil}%
      \fullrule\kern\cellwd{0}\kern\cellwd{1}\kern\cellwd{2}\kern\cellwd{3}%
      \fullrule\kern\cellwd{4}\kern\cellwd{5}\kern\cellwd{6}\kern\cellwd{7}%
      \fullrule\kern\cellwd{8}\kern\cellwd{9}\kern\cellwd{10}\kern\cellwd{11}%
      \fullrule\kern\cellwd{12}\kern\cellwd{13}\kern\cellwd{14}\kern\cellwd{15}%
      \fullrule\kern\indexwd}%
    \kern-\ht0 \kern-\dp0 \unvbox0}%
  \endgroup
}

\def\dochartA#1{%
  \begingroup
  \global\curchar=0 \global\currow=0 \global\curcol=0
  \def\hline{\kern2pt\hrule\kern3pt }%
  \setbox0\vbox{#1%
    \def\0{\hbox to\cellwd{\curcol}{\hss\charnumber\hss}}%
    \colnumbers
    \hline
    \setrow\setrow\setrow\setrow
    \hline
    \setrow\setrow\setrow\setrow
    \hline
    \setrow\setrowX\setrow\setrowX %
%    \hline %
    \setrow\setrowX\setrow\setrowX %
    \hline %
    \colnumbers
  }%
  \vbox{%
    \hbox to\hsize{\kern\indexwd
      \def\fullrule{\hfil\vrule height\ht0 depth\dp0\hfil}%
      \fullrule\kern\cellwd{0}\kern\cellwd{1}\kern\cellwd{2}\kern\cellwd{3}%
      \fullrule\kern\cellwd{4}\kern\cellwd{5}\kern\cellwd{6}\kern\cellwd{7}%
      \fullrule\kern\cellwd{8}\kern\cellwd{9}\kern\cellwd{10}\kern\cellwd{11}%
      \fullrule\kern\cellwd{12}\kern\cellwd{13}\kern\cellwd{14}\kern\cellwd{15}%
      \fullrule\kern\indexwd}%
    \kern-\ht0 \kern-\dp0 \unvbox0}%
  \endgroup
}

\def\dochartB#1{%
  \begingroup
  \global\curchar=0 \global\currow=0 \global\curcol=0
  \def\hline{\kern2pt\hrule\kern3pt }%
  \setbox0\vbox{#1%
    \def\0{\hbox to\cellwd{\curcol}{\hss\charnumber\hss}}%
    \colnumbers
    \hline
    \setrow\setrow\setrow\setrow
    \hline
    \setrow\setrow%\setrow\setrow
    \hline
    \colnumbers
  }%
  \vbox{%
    \hbox to\hsize{\kern\indexwd
      \def\fullrule{\hfil\vrule height\ht0 depth\dp0\hfil}%
      \fullrule\kern\cellwd{0}\kern\cellwd{1}\kern\cellwd{2}\kern\cellwd{3}%
      \fullrule\kern\cellwd{4}\kern\cellwd{5}\kern\cellwd{6}\kern\cellwd{7}%
      \fullrule\kern\cellwd{8}\kern\cellwd{9}\kern\cellwd{10}\kern\cellwd{11}%
      \fullrule\kern\cellwd{12}\kern\cellwd{13}\kern\cellwd{14}\kern\cellwd{15}%
      \fullrule\kern\indexwd}%
    \kern-\ht0 \kern-\dp0 \unvbox0}%
  \endgroup
}

\def\dochartC#1{%
  \begingroup
  \global\curchar=0 \global\currow=0 \global\curcol=0
  \def\hline{\kern2pt\hrule\kern3pt }%
  \setbox0\vbox{#1%
    \def\0{\hbox to\cellwd{\curcol}{\hss\charnumber\hss}}%
    \colnumbers
    \hline
    \setrow\setrow\setrow\setrow
    \hline
    \setrow\setrow\setrow\setrow
    \hline
    \setrow\setrow
    \hline
    \colnumbers
  }%
  \vbox{%
    \hbox to\hsize{\kern\indexwd
      \def\fullrule{\hfil\vrule height\ht0 depth\dp0\hfil}%
      \fullrule\kern\cellwd{0}\kern\cellwd{1}\kern\cellwd{2}\kern\cellwd{3}%
      \fullrule\kern\cellwd{4}\kern\cellwd{5}\kern\cellwd{6}\kern\cellwd{7}%
      \fullrule\kern\cellwd{8}\kern\cellwd{9}\kern\cellwd{10}\kern\cellwd{11}%
      \fullrule\kern\cellwd{12}\kern\cellwd{13}\kern\cellwd{14}\kern\cellwd{15}%
      \fullrule\kern\indexwd}%
    \kern-\ht0 \kern-\dp0 \unvbox0}%
  \endgroup
}

\def\dochartD#1{%
  \begingroup
  \global\curchar=0 \global\currow=0 \global\curcol=0
  \def\hline{\kern2pt\hrule\kern3pt }%
  \setbox0\vbox{#1%
    \def\0{\hbox to\cellwd{\curcol}{\hss\charnumber\hss}}%
    \colnumbers
    \hline
    \setrow\setrow\setrow\setrow
    \hline
    \setrow\setrow\setrow\setrow
    \hline
    \setrow\setrow\setrow\setrow
    \hline
    \setrow\setrow\setrow\setrow
    \hline
    \colnumbers
  }%
  \vbox{%
    \hbox to\hsize{\kern\indexwd
      \def\fullrule{\hfil\vrule height\ht0 depth\dp0\hfil}%
      \fullrule\kern\cellwd{0}\kern\cellwd{1}\kern\cellwd{2}\kern\cellwd{3}%
      \fullrule\kern\cellwd{4}\kern\cellwd{5}\kern\cellwd{6}\kern\cellwd{7}%
      \fullrule\kern\cellwd{8}\kern\cellwd{9}\kern\cellwd{10}\kern\cellwd{11}%
      \fullrule\kern\cellwd{12}\kern\cellwd{13}\kern\cellwd{14}\kern\cellwd{15}%
      \fullrule\kern\indexwd}%
    \kern-\ht0 \kern-\dp0 \unvbox0}%
  \endgroup
}

\def\dochartE#1{%
  \begingroup
  \global\curchar=0 \global\currow=0 \global\curcol=0
  \def\hline{\kern2pt\hrule\kern3pt }%
  \setbox0\vbox{#1%
    \def\0{\hbox to\cellwd{\curcol}{\hss\charnumber\hss}}%
    \colnumbers
    \hline
    \setrow\setrow\setrow\setrow
    \hline
    \setrow\setrow\setrow\setrow
    \hline
    \setrow\setrow\setrow\setrow
    \hline
    \setrowX\setrow\setrowX\setrow
    \hline
    \colnumbers
  }%
  \vbox{%
    \hbox to\hsize{\kern\indexwd
      \def\fullrule{\hfil\vrule height\ht0 depth\dp0\hfil}%
      \fullrule\kern\cellwd{0}\kern\cellwd{1}\kern\cellwd{2}\kern\cellwd{3}%
      \fullrule\kern\cellwd{4}\kern\cellwd{5}\kern\cellwd{6}\kern\cellwd{7}%
      \fullrule\kern\cellwd{8}\kern\cellwd{9}\kern\cellwd{10}\kern\cellwd{11}%
      \fullrule\kern\cellwd{12}\kern\cellwd{13}\kern\cellwd{14}\kern\cellwd{15}%
      \fullrule\kern\indexwd}%
    \kern-\ht0 \kern-\dp0 \unvbox0}%
  \endgroup
}
\def\colnumbers{\hbox to\hsize{\global\curcol 0
  \def\1{\hbox to\cellwd{\curcol}{\hfil\colnumber\hfil}}%
  \kern\indexwd\hfil\hfil
  \1\1\1\1\hfil\hfil \1\1\1\1\hfil\hfil
  \1\1\1\1\hfil\hfil \1\1\1\1\hfil\hfil
  \kern\indexwd}%
}

\def\dochartF#1{%
  \begingroup
  \global\curchar=0 \global\currow=0 \global\curcol=0
  \def\hline{\kern2pt\hrule\kern3pt }%
  \setbox0\vbox{#1%
    \def\0{\hbox to\cellwd{\curcol}{\hss\charnumber\hss}}%
    \colnumbers
    \hline
    \setrow\setrow\setrow\setrow
    \hline
    \setrow\setrow\setrow\setrow
    \hline
    \setrow\setrow\setrow
    \hline
    \colnumbers
  }%
  \vbox{%
    \hbox to\hsize{\kern\indexwd
      \def\fullrule{\hfil\vrule height\ht0 depth\dp0\hfil}%
      \fullrule\kern\cellwd{0}\kern\cellwd{1}\kern\cellwd{2}\kern\cellwd{3}%
      \fullrule\kern\cellwd{4}\kern\cellwd{5}\kern\cellwd{6}\kern\cellwd{7}%
      \fullrule\kern\cellwd{8}\kern\cellwd{9}\kern\cellwd{10}\kern\cellwd{11}%
      \fullrule\kern\cellwd{12}\kern\cellwd{13}\kern\cellwd{14}\kern\cellwd{15}%
      \fullrule\kern\indexwd}%
    \kern-\ht0 \kern-\dp0 \unvbox0}%
  \endgroup
}


\def\setrow{\hbox to\hsize{%
  \hbox to\indexwd{\hfil\rownumber\kern.2em}\hfil\hfil
  \0\0\0\0\hfil\hfil \0\0\0\0\hfil\hfil
  \0\0\0\0\hfil\hfil \0\0\0\0\hfil\hfil
  \hbox to\indexwd{\ttfamily\kern.2em \rownumber\hfil}}%
  \global\advance\currow 1 }%

\def\setrowX{\global\advance\curchar16\global\advance\currow 1\relax}

\def\cellwd#1{20pt}% initialize

\def\measurecolwidths#1{%
  \tempcellwd\hsize \advance\tempcellwd-2\indexwd
  \advance\tempcellwd -12pt
  \divide\tempcellwd 16
  \xdef\cellwd##1{\the\tempcellwd}%
}

\def \table #1#2#3{\par\penalty-200 \bigskip
  \font #1=#2 \relax
  \vbox{\hsize=29pc
    \measurecolwidths{#1}%
    \centerline{#3 -- {\tt#2}}%
    \medskip
    \dochart{#1}%
}}


\def \tableA #1#2#3{\par\penalty-200 \bigskip
  \font #1=#2 \relax
  \vbox{\hsize=29pc
    \measurecolwidths{#1}%
    \centerline{#3 -- {\tt#2}}%
    \medskip
    \dochartA{#1}%
}}

\def \tableB #1#2#3{\par\penalty-200 \bigskip
  \font #1=#2 \relax
  \vbox{\hsize=29pc
    \measurecolwidths{#1}%
    \centerline{#3 -- {\tt#2}}%
    \medskip
    \dochartB{#1}%
}}

\def \tableC #1#2#3{\par\penalty-200 \bigskip
  \font #1=#2 \relax
  \vbox{\hsize=29pc
    \measurecolwidths{#1}%
    \centerline{#3 -- {\tt#2}}%
    \medskip
    \dochartC{#1}%
}}

\def \tableD #1#2#3{\par\penalty-200 \bigskip
  \font #1=#2 \relax
  \vbox{\hsize=29pc
    \measurecolwidths{#1}%
    \centerline{#3 -- {\tt#2}}%
    \medskip
    \dochartD{#1}%
}}

\def \tableE #1#2#3{\par\penalty-200 \bigskip
  \font #1=#2 \relax
  \vbox{\hsize=29pc
    \measurecolwidths{#1}%
    \centerline{#3 -- {\tt#2}}%
    \medskip
    \dochartE{#1}%
}}

\def \tableF #1#2#3{\par\penalty-200 \bigskip
  \font #1=#2 \relax
  \vbox{\hsize=29pc
    \measurecolwidths{#1}%
    \centerline{#3 -- {\tt#2}}%
    \medskip
    \dochartF{#1}%
}}

\makeatother

\begin{document}

\title{The \texttt{TX} Fonts%
\thanks{Special thanks to those who reported problems of
\texttt{TX} fonts and provided suggestions!}}

\author{Young Ryu}

\date{December 15, 2000}

\maketitle

\tableofcontents

\clearpage
\section{Introduction}

The \texttt{TX} fonts consist of
\begin{enumerate}\itemsep=0pt
\item virtual text roman fonts using Adobe Times (or URW NimbusRomNo9L) with
      some modified and additional text symbols in OT1, T1, TS1, and LY1 encodings
\item \textsf{virtual text sans serif fonts using Adobe Helvetica (or URW NimbusSanL) with
      additional text symbols in OT1, T1, TS1, and LY1 encodings}
\item \texttt{monospaced typewriter fonts in OT1, T1, TS1, and LY1 encodings}
\item math alphabets using Adobe Times (or URW NimbusRomNo9L)
      with modified metrics
\item math fonts of all symbols corresponding to those of Computer Modern
      math fonts (CMSY, CMMI, CMEX, and Greek letters of CMR)
\item math fonts of all symbols corresponding to those of AMS fonts
      (MSAM and MSBM)
\item additional math fonts of various symbols
\end{enumerate}
%
All fonts are in the Type 1 format (in \texttt{afm} and \texttt{pfb} files).
Necessary \texttt{tfm} and \texttt{vf} files together with
\LaTeXe\ package files and font map files for \texttt{dvips} are
provided.

\begin{bfseries}%\itshape
The \texttt{TX} fonts and related files are distributed 
without any guaranty or warranty.
I do not assume responsibility for any actual or possible
damages or losses, directly or indirectly caused by the
distributed files.
\end{bfseries}
The \texttt{TX} fonts are distributed under the GNU public license (GPL)\@.
The fonts will be improved and additional glyphs will be added
in the future.

\section{Changes}

\begin{description}
\item[1.0] (October 25, 2000) 1st public release
\item[2.0] (November 2, 2000)
     \begin{itemize}
     \item An encoding error in txi and txbi (`\textdollar' \texttt{"24}) is fixed.
     \item Mistakes in symbol declarations for `\AA' and `\aa' in \texttt{txfonts.sty}
           are fixed.
     \item $\lambda$ (\texttt{"15} of txmi and txbmi),
           $\lambdaslash$ (\texttt{"6E} of txsyc and txbsyc), and
	 $\lambdabar$ (\texttt{"6F} of txsyc and txbsyc)
	 are updated to be more slanted.
     \item More symbols added in txexa and txbexa (\texttt{"29}--\texttt{"2E})
           and in txsyc and txbsyc (\texttt{"80}--\texttt{"94}).
     \item Some fine tuning of a few glyphs.
     \item Math italic font metrics are improved.
     \item Text font metrics are improved.
     \item T1 and TS1 encodings are supported. (Not all TS1 encoding glyphs are implemented.)
     \end{itemize}
\item[2.1] (November 18, 2000)
     \begin{itemize}
     \item Complete implementation of TS1 encoding fonts.
     \item Various improvements of font metrics and font encodings. For instance,
           the bogus entry of char \texttt{'27} in T1 encoding virtual font files
           are removed. (This bogus entry caused ``warning char 23 replaced
           by \bs.notdef'' with PDF\TeX/PDF\LaTeX.)
     \item Helvetica-based TX sans serif fonts in OT1, T1, and TS1 encodings.
     \item Monospaced TX typewriter fonts, which are thicker than Courier (and thus may look better
           with Times), in OT1, T1, and TS1 encodings.
     \end{itemize}
\item[2.2] (November 22, 2000)
     \begin{itemize}
     \item LY1 encoding support
     \item Monospaced typewriter fonts redone (Uppercase letters are tall enough to match with Times.)
     \item Various glyph and metric improvement
     \end{itemize}
\item[2.3] (December 6, 2000)
     \begin{itemize}
     \item Math extension fonts (radical symbols) updated
     \item Alternative blackboard bold letters ($\varmathbb{A}\ldots\varmathbb{Z}$ and $\varBbbk$)
           introduced. (Enter \verb|$\varmathbb{...}$| and \verb|$\varBbk$| to get them.)
     \item More large operators symbols
     \item Now \verb|\lbag| ($\lbag$) and \verb|\rbag| ($\rbag$) are
          delimiters.
     \item Alternative math alphabets $\varg$ and $\vary$ added
     \end{itemize}
\item[2.4] (December 12, 2000)
     \begin{itemize}
     \item An encoding mistake in text companion typewriter fonts fixed
     \item Bugs in \LaTeX\ input files fixed
     \end{itemize}
\item[3.0] (December 14, 2000)
     \begin{itemize}
     \item Minor problem fixes.
     \end{itemize}
\item[3.1] (December 15, 2000)
     \begin{itemize}
     \item Alternative math alphabets $\varv$ and $\varw$ added
     \item Hopefully, this is the final release \ldots
     \end{itemize}
\end{description}

\section{A Problem: \texttt{DVIPS} Partial Font Downloading}

It was reported that when \texttt{TX} fonts
are partially downloaded with \texttt{dvips},
some HP Laserprinters (with Postscript) cannot
print documents. To resolve this problem,
turn the partial font downloading off.
See the \texttt{dvips} document for various ways to
turn off partial font downloading.

\textbf{\itshape Even though one does not observe such a problem,
I would like to strongly recommend to turn off \texttt{dvips}
partial font downloading.}
%I think the \texttt{dvips} partial font downloading
%mechanism appears to have some problems. For instance,
%when Adobe Times fonts are set to be downloaded, e.g.,
%\begin{verbatim}
%   ptmr8r Times-Roman "TeXBase1Encoding ReEncodeFont" <8r.enc <tir_____.pfb
%\end{verbatim}
%\TeX ing \texttt{testfont.tex} on \texttt{ptmr8r}
%and \texttt{dvips}ing \texttt{testfont.dvi}
%with partial font download on give
%\begin{verbatim}
%   WARNING: Not all chars found
%\end{verbatim}
%This specific warning seems to be harmless. But,
%in my opinion, this should not happen.


\section{Installation}

Put all files in \texttt{afm}, \texttt{tfm}, \texttt{vf},
and \texttt{pfb} files in proper locations of your \TeX\ system.
For Mik\TeX, they may go
\begin{verbatim}
     \localtexmf\fonts\afm\txr\
     \localtexmf\fonts\tfm\txr\
     \localtexmf\fonts\vf\txr\
     \localtexmf\fonts\type1\txr\
\end{verbatim}
All files of the \texttt{input} directory must
be placed where \LaTeX\ finds its package files.
For Mik\TeX, they may go
\begin{verbatim}
     \localtexmf\tex\latex\txr\
\end{verbatim}
Put the \texttt{txr.map}, \texttt{txr1.map}, \texttt{txr2.map}, and \texttt{tx8r.enc}%
\footnote{The \texttt{tx8r.enc} file is identical to \texttt{8r.enc}.
I included \texttt{tx8r.enc} because (1)~some \TeX\ installation might not have \texttt{8r.enc} and
(2)~including \texttt{8r.enc} would result in multiple copies of \texttt{8r.enc} for \TeX\ systems
that already have it.}
files of the \texttt{dvips}
directory in a proper place that \texttt{dvips} refers to.
For Mik\TeX, they may go
\begin{verbatim}
     \localtexmf\dvips\config\
\end{verbatim}
Also add the reference to \texttt{txr.map} in
the \texttt{dvips} configuration file (\texttt{config.ps})
\begin{verbatim}
     . . .
     % Configuration of postscript type 1 fonts:
     p psfonts.map
     p +txr.map
     . . .
\end{verbatim}
and in the PDF\TeX\ configuration file (\texttt{pdftex.cfg})
\begin{verbatim}
     . . .
     % pdftex.map is set up by texmf/dvips/config/updmap
     map pdftex.map
     map +txr.map
     . . .
\end{verbatim}
(The \texttt{txr.map} file has only named references to the Adobe Times fonts;
the \texttt{txr1.map} file makes \texttt{dvips} load Adobe Times font files;
and the \texttt{txr2.map}  file makes \texttt{dvips} load URW NimbusRomNo9L font files.)

\section{Using the \texttt{TX} Fonts with \LaTeX}

It is as simple as
\begin{verbatim}
     \documentclass{article}
     \usepackage{txfonts}
     %\normalfont                     % Just in case ...
     %\usepackage[T1]{fontenc}        % To use T1 encoding fonts
     %\usepackage[LY1]{fontenc}       % To use LY1 encoding fonts
     %\usepackage{textcomp}           % To use text companion fonts

     \begin{document}

     This is a very short article.

     \end{document}
\end{verbatim}

The standard \LaTeX\ distribution does not include
files supporting the LY1 encoding.
One needs at least \texttt{ly1enc.def}, which is available
from both CTAN and Y\&Y (\texttt{www.yandy.com}).
At the time this document was written, CTAN had
an old version (1997/03/21 v0.3); \texttt{ly1enc.def}
available from Y\&Y's downloads sites was dated on 1998/04/21 v0.4.

\section{Additional Symbols in the \texttt{TX} Math Fonts}

\emph{All} CM symbols are included in the \texttt{TX} math fonts.
In addition, the \texttt{TX} math fonts provide or modify
the following symbols, including all of AMS and most of \LaTeX\ symbols.

\subsubsection*{Binary Operator Symbols}
\begin{eqnarray*}[c@{\enskip}l@{\qquad\qquad\qquad}c@{\enskip}l@{\qquad\qquad\qquad}c@{\enskip}l]
\Ent{medcirc}&
\Ent{medbullet}&
\Ent{invamp}\\
\Ent{circledwedge}&
\Ent{circledvee}&
\Ent{circledbar}\\
\Ent{circledbslash}&
\Ent{nplus}&
\Ent{boxast}\\
\Ent{boxbslash}&
\Ent{boxbar}&
\Ent{boxslash}\\
\Ent{Wr}&
\Ent{sqcupplus}&
\Ent{sqcapplus}\\
\Ent{rhd}&
\Ent{lhd}&
\Ent{unrhd}\\
\Ent{unlhd}
\end{eqnarray*}

\subsubsection*{Binary Relation Symbols}
\begin{eqnarray*}[c@{\enskip}l@{\quad}c@{\enskip}l@{\quad}c@{\enskip}l]
\Ent{mappedfrom}&
\Ent{longmappedfrom}&
\Ent{Mapsto}\\
\Ent{Longmapsto}&
\Ent{Mappedfrom}&
\Ent{Longmappedfrom}\\
\Ent{mmapsto}&
\Ent{longmmapsto}&
\Ent{mmappedfrom}\\
\Ent{longmmappedfrom}&
\Ent{Mmapsto}&
\Ent{Longmmapsto}\\
\Ent{Mmappedfrom}&
\Ent{Longmmappedfrom}&
\Ent{varparallel}\\
\Ent{varparallelinv}&
\Ent{nvarparallel}&
\Ent{nvarparallelinv}\\
\Ent{colonapprox}&
\Ent{colonsim}&
\Ent{Colonapprox}\\
\Ent{Colonsim}&
\Ent{doteq}&
\Ent{multimapinv}\\
\Ent{multimapboth}&
\Ent{multimapdot}&
\Ent{multimapdotinv}\\
\Ent{multimapdotboth}&
\Ent{multimapdotbothA}&
\Ent{multimapdotbothB}\\
\Ent{VDash}&
\Ent{VvDash}&
\Ent{cong}\\
\Ent{preceqq}&
\Ent{succeqq}&
\Ent{nprecsim}\\
\Ent{nsuccsim}&
\Ent{nlesssim}&
\Ent{ngtrsim}\\
\Ent{nlessapprox}&
\Ent{ngtrapprox}&
\Ent{npreccurlyeq}\\
\Ent{nsucccurlyeq}&
\Ent{ngtrless}&
\Ent{nlessgtr}\\
\Ent{nbumpeq}&
\Ent{nBumpeq}&
\Ent{nbacksim}\\
\Ent{nbacksimeq}&
\EEnt{neq}{ne}&
\Ent{nasymp}\\
\Ent{nequiv}&
\Ent{nsim}&
\Ent{napprox}\\
\Ent{nsubset}&
\Ent{nsupset}&
\Ent{nll}\\
\Ent{ngg}&
\Ent{nthickapprox}&
\Ent{napproxeq}\\
\Ent{nprecapprox}&
\Ent{nsuccapprox}&
\Ent{npreceqq}\\
\Ent{nsucceqq}&
\Ent{nsimeq}&
\Ent{notin}\\
\EEnt{notni}{notowns}&
\Ent{nSubset}&
\Ent{nSupset}\\
\Ent{nsqsubseteq}&
\Ent{nsqsupseteq}&
\Ent{coloneqq}\\
\Ent{eqqcolon}&
\Ent{coloneq}&
\Ent{eqcolon}\\
\Ent{Coloneqq}&
\Ent{Eqqcolon}&
\Ent{Coloneq}\\
\Ent{Eqcolon}&
\Ent{strictif}&
\Ent{strictfi}\\
\Ent{strictiff}&
\Ent{circledless}&
\Ent{circledgtr}\\
\Ent{lJoin}&
\Ent{rJoin}&
\EEnt{Join}{lrJoin}\\
\Ent{openJoin}&
\Ent{lrtimes}&
\Ent{opentimes}\\
\Ent{nsqsubset}&
\Ent{nsqsupset}&
\Ent{dashleftarrow}\\
%\EEnt{dashrightarrow}{dasharrow}&
\Ent{dashrightarrow}&
\Ent{dashleftrightarrow}&
\Ent{leftsquigarrow}\\
\Ent{ntwoheadrightarrow}&
\Ent{ntwoheadleftarrow}&
\Ent{Nearrow}\\
\Ent{Searrow}&
\Ent{Nwarrow}&
\Ent{Swarrow}\\
\Ent{Perp}&
\Ent{leadstoext}&
\Ent{leadsto}\\
\Ent{boxright}&
\Ent{boxleft}&
\Ent{boxdotright}\\
\Ent{boxdotleft}&
\Ent{Diamondright}&
\Ent{Diamondleft}\\
\Ent{Diamonddotright}&
\Ent{Diamonddotleft}&
\Ent{boxRight}\\
\Ent{boxLeft}&
\Ent{boxdotRight}&
\Ent{boxdotLeft}\\
\Ent{DiamondRight}&
\Ent{DiamondLeft}&
\Ent{DiamonddotRight}\\
\Ent{DiamonddotLeft}&
\Ent{circleright}&
\Ent{circleleft}\\
\Ent{circleddotright}&
\Ent{circleddotleft}&
\Ent{multimapbothvert}\\
\Ent{multimapdotbothvert}&
\Ent{multimapdotbothAvert}&
\Ent{multimapdotbothBvert}
\end{eqnarray*}

\subsubsection*{Ordinary Symbols}
\begin{eqnarray*}[c@{\enskip}l@{\qquad\qquad\qquad}c@{\enskip}l@{\qquad\qquad\qquad}c@{\enskip}l]
\Ent{alphaup}&
\Ent{betaup}&
\Ent{gammaup}\\
\Ent{deltaup}&
\Ent{epsilonup}&
\Ent{varepsilonup}\\
\Ent{zetaup}&
\Ent{etaup}&
\Ent{thetaup}\\
\Ent{varthetaup}&
\Ent{iotaup}&
\Ent{kappaup}\\
\Ent{lambdaup}&
\Ent{muup}&
\Ent{nuup}\\
\Ent{xiup}&
\Ent{piup}&
\Ent{varpiup}\\
\Ent{rhoup}&
\Ent{varrhoup}&
\Ent{sigmaup}\\
\Ent{varsigmaup}&
\Ent{tauup}&
\Ent{upsilonup}\\
\Ent{phiup}&
\Ent{varphiup}&
\Ent{chiup}\\
\Ent{psiup}&
\Ent{omegaup}&
\Ent{Diamond}\\
\Ent{Diamonddot}&
\Ent{Diamondblack}&
\Ent{lambdaslash}\\
\Ent{lambdabar}&
\Ent{varclubsuit}&
\Ent{vardiamondsuit}\\
\Ent{varheartsuit}&
\Ent{varspadesuit}&
\Ent{Top}\\
\Ent{Bot}
\end{eqnarray*}

\subsubsection*{Math Alphabets}

\begin{eqnarray*}[c@{\enskip}l@{\qquad\qquad\qquad}c@{\enskip}l@{\qquad\qquad\qquad}c@{\enskip}l@{\qquad\qquad\qquad}c@{\enskip}l]
\Ent{varg} &
\Ent{varv} &
\Ent{varw} &
\Ent{vary}
\end{eqnarray*}
In order to replace math alphabets $g$, $v$, $w$, and $y$ by these alternatives,
use the \texttt{varg} option with the \texttt{txfonts} package:
\begin{verbatim}
     \usepackage[varg]{txfonts}
\end{verbatim}
Then, \verb|$g$|, \verb|$v$|, \verb|$w$|, and \verb|$y$| will produce
these $\varg$, $\varv$, $\varw$, and~$\vary$ (instead of $g$, $v$, $w$, and~$y$).
%
Notice that $\varv$ (the alternative \textit{v})
is more clearly distingiushed from $\nu$ (the lowercase Greek nu).
However, this is not without cost:
it looks similar to $\upsilon$ (the lowercase Greek upsilon).

%\footnote{A comment on Times New Roman fonts:
%the italic \textit{v} of Times New Roman (both Type 1 and TrueType versions),
%but not that of Times, is very badly designed. The starting serif at the left-top
%corner of the letter is very different from other letters' corresponding portion.
%However, that of Times New Roman bold italic is consistent with others.
%Further, in the TrueType version of Times New Roman italic and bold italic,
%the lowercase Greek $\nu$ (nu) is exactly same as \textit{v} (i.e., linked to \textit{v}).
%For the mathematical typesetting purpose, this is undesirable.
%In \texttt{TX} fonts, the lowercase Greek $\nu$ (nu) is not identical to
%\textit{v}, but very similar. The alternative $\varv$ is provided to
%be more clearly distingiushed from the lowercase Greek $\nu$ (nu).
%The alternative $\varw$ is provided to ensure consistency.}

\subsubsection*{Large Operator Symbols}

\begin{eqnarray*}[c@{\enskip}l@{\quad}c@{\enskip}l@{\quad}c@{\enskip}l]
\Ent{bignplus}&
\Ent{bigsqcupplus}&
\Ent{bigsqcapplus}\\
\Ent{bigsqcap}&
\Ent{bigsqcap}&
\Ent{varprod}\\
\Ent{oiint}&
\Ent{oiiint}&
\Ent{ointctrclockwise}\\
\Ent{ointclockwise}&
\Ent{varointctrclockwise}&
\Ent{varointclockwise}\\
\Ent{sqint}&
\Ent{sqiintop}&
\Ent{sqiiintop}\\
\Ent{fint}&
\Ent{iint}&
\Ent{iiint}\\
\Ent{iiiint}&
\Ent{idotsint}&
\Ent{oiintctrclockwise}\\
\Ent{oiintclockwise}&
\Ent{varoiintctrclockwise}&
\Ent{varoiintclockwise}\\
\Ent{oiiintctrclockwise}&
\Ent{oiiintclockwise}&
\Ent{varoiiintctrclockwise}\\
\Ent{varoiiintclockwise}&
\end{eqnarray*}

\subsubsection*{Delimiters}
\begin{eqnarray*}[c@{\enskip}l@{\qquad\qquad\qquad}c@{\enskip}l@{\qquad\qquad\qquad}c@{\enskip}l@{\qquad\qquad\qquad}c@{\enskip}l]
\Big\llbracket&\texttt{\bs llbracket}&
\Big\rrbracket&\texttt{\bs rrbracket}&
\Big\lbag&\texttt{\bs lbag}&
\Big\rbag&\texttt{\bs rbag}
\end{eqnarray*}

%\subsubsection*{Parentheses}
%\begin{eqnarray*}[c@{\enskip}l@{\qquad\qquad\qquad}c@{\enskip}l@{\qquad\qquad\qquad}c@{\enskip}l@{\qquad\qquad\qquad}c@{\enskip}l]
%\Ent{lbag}&
%\Ent{rbag}&
%\Ent{Lbag}&
%\Ent{Rbag}
%\end{eqnarray*}

\subsubsection*{Miscellaneous}

\verb|$\mathfrak{...}$| produces
$\mathfrak{A} \ldots \mathfrak{Z}$ and $\mathfrak{a} \ldots \mathfrak{z}$.
\verb|$\varmathbb{...}$| produces
$\varmathbb{A} \ldots \varmathbb{Z}$ (lowercase letters only);
\verb|\varBbbk| produces $\varBbbk$.
Note that the \AmS\ math font command \verb|$\mathbb{...}$| produces
$\mathbb{A} \ldots \mathbb{Z}$;
\verb|\Bbbk| produces $\Bbbk$.
If you find the alternative blackboard letters are better, then do
\begin{verbatim}
   \let\mathbb=\varmathbb
   \let\Bbbk=\varBbbk
\end{verbatim}


\section{Remarks}

\subsection{Some Font Design Issues}

The Adobe Times fonts are thicker than the CM fonts.
Designing math fonts for Times based on the rule thickness of Times
`$=$', `$-$', `$+$', `/', `$<$', etc.\
would result in too thick math symbols,
in my opinion.\footnote{I have designed many math symbols
(corresponding to those in CMMI and CMSY)
based on the rule thickness of original Times `$=$', etc.
At that time, I noticed that the symbols,
especially some bold math symbols, are extremely thick.
Perhaps, in the future, I will complete all math symbols
based on the rule thickness of original Times `$=$', etc.\
and release in public, so that users will judge
whether they are acceptable or not~\ldots.}
In the \texttt{TX} fonts, these glyphs are thinner
than those of original Times fonts. That is, the rule thickness
of these glyphs is around 85\% of that of the Times fonts,
but still thicker than that of the CM fonts.

For negated relation symbols, the CM fonts composes
relation symbols with the negation slash (\texttt{"36} in CMSY).
Even though the CM fonts were very carefully designed
to look reasonable when negated relation symbols are composed
(except `$\notin$' \verb|\notin|, which is composed of
`$\in$' and the normal slash `$/$'),
the AMS font set includes many negated relation symbols,
mainly because the vertical placement and
height\slash depth of the negation slash are not optimal
when composed with certain relation symbols, I guess.
The \texttt{TX} fonts include the negation slash symbol
(\texttt{"36} in txsy), which could be composed with
relation symbols to give reasonably looking negated related symbols.
I believe, however, explicitly designed negated relation symbols
are looking better than composed relation symbols.
Thus, in addition to negated relation symbols matching those of
the AMS fonts, many negated symbols such as `$\neq$' are introduced
in the \texttt{TX} fonts.

Further, in order to maintain editing compatibility with
vanilla \LaTeXe\ typesetting, \verb|\not| is redefined in \texttt{txfonts.sty} 
so that when \verb|\not\XYZ| is processed,
if \verb|\notXYZ| or \verb|\nXYZ| is defined, it will be used
in place of \verb|\not\XYZ|; otherwise,
\verb|\XYZ| is composed with the negation slash.
For instance, `$\nprecsim$' is available as \verb|\nprecsim| in the \texttt{TX} fonts.
Thus, if \verb|\not\precsim| is typed in the document,
the \verb|\nprecsim| symbol, instead of \verb|\precsim| composed
with the negation slash, is printed.

\subsection{Times vs.\ Times New Roman}

The recent version of Acrobat is shipped with
Times New Roman instead of Times fonts.
Times New Roman fonts' italic letters (e.g.,
`\textit{A}') are substantially different from those
of Times fonts. Thus, when documents with the \texttt{TX} fonts
are processed with Acrobat, accents may not be correctly placed.
If this is a noticeable problem, use the NimbusRomNo9L fonts
(included in the Ghostscript distribution) with the \texttt{TX} fonts
through \texttt{txr2.map}.

\subsection{PDF\TeX/PDF\LaTeX\ and Standard Postscript Fonts}

PDF\TeX/PDF\LaTeX\ does not handle slanting of fonts not embedded
in the document.
Note, in the standard setup, PDF\TeX/PDF\LaTeX\ does not embed
the 14 standard Postscript fonts (Times $\times$~4,
Helvetica $\times$~4, Courier $\times$~4, Symbol, and ZapfDingbats).
As the result, PDF\TeX/PDF\LaTeX\ issues warning
(and may try to generate and use bitmapped fonts for these fonts).
If it is not desirable, a solution would be to use URW NimbusRomNo9L
and NimbusSanL fonts
which are an Adobe Times and Helvetica fonts clone. That is, in the
PDF\TeX/PDF\LaTeX\ configuration file (\texttt{pdftex.cfg}),
put \texttt{txr2.map} instead of \texttt{txr.map}
\begin{verbatim}
     . . .
     % pdftex.map is set up by texmf/dvips/config/updmap
     map pdftex.map
     map +txr2.map
     . . .
\end{verbatim}
Be sure to properly install URW NimbusRomNo9L and NimbusSanL fonts (which are
included in the Ghostscript distribution) in your texmf tree.

If you have Adobe Times and Helvetica font files, and want to embed them
in your PDF document file, do the following trick to fool PDF\TeX/PDF\LaTeX.
\begin{enumerate}\itemsep=0pt%\parskip=0pt
\item Copy \texttt{txr1.map} in the dvips configuration directory
      to \texttt{txrpdf.map} in the PDF\TeX/PDF\LaTeX\ configuration directory.
\item Edit txrpdf.map and have
\begin{small}
\begin{verbatim}
rtxptmb "TeXBase1Encoding ReEncodeFont" <tx8r.enc <tib_____.pfb
rtxptmbo ".167 SlantFont TeXBase1Encoding ReEncodeFont" <tx8r.enc <tib_____.pfb
rtxptmbi "TeXBase1Encoding ReEncodeFont" <tx8r.enc <tibi____.pfb
rtxptmr "TeXBase1Encoding ReEncodeFont" <tx8r.enc <tir_____.pfb
rtxptmro ".167 SlantFont TeXBase1Encoding ReEncodeFont" <tx8r.enc <tir_____.pfb
rtxptmri "TeXBase1Encoding ReEncodeFont" <tx8r.enc <tii_____.pfb
. . .
\end{verbatim}
\end{small}
instead of
\begin{small}
\begin{verbatim}
rtxptmb Times-Bold "TeXBase1Encoding ReEncodeFont" <tx8r.enc <tib_____.pfb
. . .
. . .
\end{verbatim}
\end{small}
Note, the actual standard Postscript fonts names such as \texttt{"Times-Bold"}
are removed. As the result, PDF\TeX/PDF\LaTeX\ will embed these standard
Postscript fonts and there will be no warning for slanting them.
\item Put \texttt{txrpdf.map} in the PDF\TeX/PDF\LaTeX\ configuration
      file (\texttt{pdftex.cfg}).
\begin{small}
\begin{verbatim}
. . .
% pdftex.map is set up by texmf/dvips/config/updmap
map pdftex.map
map +txrpdf.map
. . .
\end{verbatim}
\end{small}
\end{enumerate}

\subsection{Glyph Hinting}

The hinting of the \texttt{TX} fonts is far from ideal.
As a result, when documents with the \texttt{TX} fonts
are \emph{viewed} with Gsview (or Ghostview), you might notice
some display quality problem. When they are \emph{viewed}
with Acrobat, they look much better.
However, when they are \emph{printed} in laser printers,
there will be no quality problem.
(Note, hinting is to improve display quality on low resolution devices such as
display screens.)

\subsection{Glyphs in Low Positions}

It is known that Acrobat often does not properly handle
CM font glyphs placed between \texttt{"00} and \texttt{"1F}.
Thus, most Type 1 versions of CM fonts publicly available
have these glyphs in higher positions above \texttt{"7F}.
When the \texttt{-G} flag is used with \texttt{dvips},
those glyphs in low positions are shifted to higher positions.
The \texttt{TX} text fonts have
glyphs in the low positions between \texttt{"00} and \texttt{"1F}.
As of now, these glyphs are not available in higher positions above \texttt{"7F}.
Thus, when run \texttt{dvips}, do not use the \texttt{-G}
flag (or remove \texttt{G} in the \texttt{dvips} configuration file).
Especially, do not use \texttt{config.pdf}.
In my computer systems, Acrobat correctly handles glyphs in low positions.
However, if this known Acrobat problem occurs in other computer systems,
I will modify the \texttt{TX} fonts so that glyphs in low positions
are also available in higher positions.

\section{Font Charts}

The original Computer Modern (CM) text fonts (aka \TeX\ text fonts)
have the OT1 encoding. The OT1 \texttt{TX} text fonts follow
the CM fonts' encoding as much as possible, but have some
variations and additions:
\begin{itemize}\parskip=0pt\itemsep=0pt
\item The position \texttt{"24} of text italic fonts has
      the dollar symbol (\textit{\textdollar}), not the sterling symbol (\textit{\textsterling}).
\item The uppercase and lowercase lslash (\L, \l) and aring (\AA, \aa) letters are added.
\item The cent (\ifx\textcentoldstyle\undefined\textcent\else\textcentoldstyle\fi)
      and sterling (\textsterling) symbols are added.
\end{itemize}
The original CM text fonts have somewhat different encodings in
\textsc{cap \& small cap} and \texttt{typewriter} fonts.
\texttt{TX} fonts corresponding to them have the original CM encodings,
not the strict OT1 encoding.

The T1 encoding text fonts (known as EC fonts) are designed to
replace the CM text fonts in the OT1 encoding.
The LY1 encoding is another text font encoding, which is based
on both \TeX\ and ANSI encodings.
Both T1 and LY1 encoding fonts are especially useful to typeset
European languages with proper hyphenation.
The TS1 encoding text companion fonts (known as TC fonts) have
additional text symbols.
All corresponding \texttt{TX} fonts are implemented.

The Computer Modern (CM) math fonts (aka \TeX\ math fonts)
consist of three fonts: math italic (CMMI), math symbols (CMSY), and
math extension (CMEX). The American Mathematical Society provided
two additional math symbol fonts (MSAM and MSBM).
The \texttt{TX} math fonts include those exactly corresponding to them.
In addition, the \texttt{TX} math fonts include math italic A,
math symbols C, and math extension A fonts.


\subsection{OT1 (CM) Encoding Text Fonts}

These fonts' encodings are identical to those of corresponding CM fonts,
except 6~additional glyphs.

\begin{center}
\centering
\leavevmode\hbox{\tableA \fonttab{txr}{Text Roman Upright}}

\bigskip\bigskip
\leavevmode\hbox{\tableA \fonttab{txi}{\textit{Text Roman Italic}}}

\bigskip\bigskip
\leavevmode\hbox{\tableA \fonttab{txsl}{\textsl{Text Roman Slanted}}}

\bigskip\bigskip
\leavevmode\hbox{\tableA \fonttab{txsc}{\textsc{Text Roman Cap \& Small Cap}}}

\bigskip\bigskip
\leavevmode\hbox{\tableA \fonttab{txss}{\textsf{Text Sans Serif Upright}}}

\bigskip\bigskip
\leavevmode\hbox{\tableA \fonttab{txsssl}{\textsf{\slshape Text Sans Serif Slanted}}}

\bigskip\bigskip
\leavevmode\hbox{\tableA \fonttab{txsssc}{\textsf{\scshape Text Sans Serif Cap \& Small Cap}}}

\bigskip\bigskip
\leavevmode\hbox{\tableA \fonttab{txtt}{\texttt{Text Typewriter Upright}}}

\bigskip\bigskip
\leavevmode\hbox{\tableA \fonttab{txttsl}{\texttt{\slshape Text Typewriter Slanted}}}

\bigskip\bigskip
\leavevmode\hbox{\tableA \fonttab{txttsc}{\texttt{\scshape Text Typewriter Cap \& Small Cap}}}
\end{center}

\subsection{T1 (EC) Cork Encoding Text Fonts}

These fonts' encodings are identical to those of corresponding EC fonts.

\begin{center}
\centering
\leavevmode\hbox{\tableD \fonttab{t1xr}{Text Roman Upright}}

\bigskip\bigskip
\leavevmode\hbox{\tableD \fonttab{t1xi}{\textit{Text Roman Italic}}}

\bigskip\bigskip
\leavevmode\hbox{\tableD \fonttab{t1xsl}{\textsl{Text Roman Slanted}}}

\bigskip\bigskip
\leavevmode\hbox{\tableD \fonttab{t1xsc}{\textsc{Text Roman Cap \& Small Cap}}}

\bigskip\bigskip
\leavevmode\hbox{\tableD \fonttab{t1xss}{\textsf{Text Sans Serif Upright}}}

\bigskip\bigskip
\leavevmode\hbox{\tableD \fonttab{t1xsssl}{\textsf{\slshape Text Sans Serif Slanted}}}

\bigskip\bigskip
\leavevmode\hbox{\tableD \fonttab{t1xsssc}{\textsf{\scshape Text Sans Serif  Cap \& Small Cap}}}

\bigskip\bigskip
\leavevmode\hbox{\tableD \fonttab{t1xtt}{\texttt{Text Typewriter Upright}}}

\bigskip\bigskip
\leavevmode\hbox{\tableD \fonttab{t1xttsl}{\texttt{\slshape Text Typewriter Slanted}}}

\bigskip\bigskip
\leavevmode\hbox{\tableD \fonttab{t1xttsc}{\texttt{\scshape Text Typewriter Cap \& Small Cap}}}
\end{center}

\subsection{LY1 \TeX\ and ANSI Encoding Text Fonts}

\begin{center}
\centering
\leavevmode\hbox{\tableD \fonttab{tyxr}{Text Roman Upright}}

\bigskip\bigskip
\leavevmode\hbox{\tableD \fonttab{tyxi}{\textit{Text Roman Italic}}}

\bigskip\bigskip
\leavevmode\hbox{\tableD \fonttab{tyxsl}{\textsl{Text Roman Slanted}}}

\bigskip\bigskip
\leavevmode\hbox{\tableD \fonttab{tyxsc}{\textsc{Text Roman Cap \& Small Cap}}}

\bigskip\bigskip
\leavevmode\hbox{\tableD \fonttab{tyxss}{\textsf{Text Sans Serif Upright}}}

\bigskip\bigskip
\leavevmode\hbox{\tableD \fonttab{tyxsssl}{\textsf{\slshape Text Sans Serif Slanted}}}

\bigskip\bigskip
\leavevmode\hbox{\tableD \fonttab{tyxsssc}{\textsf{\scshape Text Sans Serif Cap \& Small Cap}}}

\bigskip\bigskip
\leavevmode\hbox{\tableD \fonttab{tyxtt}{\texttt{Text Typewriter Upright}}}

\bigskip\bigskip
\leavevmode\hbox{\tableD \fonttab{tyxttsl}{\texttt{\slshape Text Typewriter Slanted}}}

\bigskip\bigskip
\leavevmode\hbox{\tableD \fonttab{tyxttsc}{\texttt{\scshape Text Typewriter Cap \& Small Cap}}}
\end{center}

\subsection{TS1 (TC) Encoding Text Companion Fonts}

These fonts' encodings are identical to those of corresponding TC fonts.

\begin{center}
\centering
\leavevmode\hbox{\tableE \fonttab{tcxr}{Text Companion Roman Upright}}

\bigskip\bigskip
\leavevmode\hbox{\tableE \fonttab{tcxi}{\textit{Text Companion Roman Italic}}}

\bigskip\bigskip
\leavevmode\hbox{\tableE \fonttab{tcxsl}{\textsl{Text Companion Roman Slanted}}}

\bigskip\bigskip
\leavevmode\hbox{\tableE \fonttab{tcxss}{\textsf{Text Companion Sans Serif Upright}}}

\bigskip\bigskip
\leavevmode\hbox{\tableE \fonttab{tcxsssl}{\textsf{\slshape Text Companion Sans Serif Slanted}}}

\bigskip\bigskip
\leavevmode\hbox{\tableE \fonttab{tcxtt}{\texttt{Text Companion Typewriter Upright}}}

\bigskip\bigskip
\leavevmode\hbox{\tableE \fonttab{tcxttsl}{\texttt{\slshape Text Companion Typewriter Slanted}}}
\end{center}

\subsection{Math Fonts}

These fonts' encodings are identical to those of corresponding CM 
and AMS Math fonts.
Additional math fonts are provided.

\begin{center}
\centering
\leavevmode\hbox{\table \fonttab{txmi}{Math Italic (Corresponding to CMMI)}}

\bigskip\bigskip
\leavevmode\hbox{\table \fonttab{txmi1}{Math Italic (Corresponding to CMMI) used with the \texttt{varg} option}}

\bigskip\bigskip
\leavevmode\hbox{\tableF \fonttab{txmia}{Math Italic A}}

\bigskip\bigskip
\leavevmode\hbox{\table \fonttab{txsy}{Math Symbols (Corresponding to CMSY)}}

\bigskip\bigskip
\leavevmode\hbox{\table \fonttab{txsya}{Math Symbols A (Corresponding to MSAM)}}

\bigskip\bigskip
\leavevmode\hbox{\table \fonttab{txsyb}{Math Symbols B (Corresponding to MSBM)}}

\bigskip\bigskip
\leavevmode\hbox{\tableC \fonttab{txsyc}{Math Symbols C}}

\bigskip\bigskip
\leavevmode\hbox{\table \fonttab{txex}{Math Extension (Corresponding to CMEX)}}

\bigskip\bigskip
\leavevmode\hbox{\tableB \fonttab{txexa}{Math Extension A}}
\end{center}

Bold versions of all fonts are available.


\end{document}
