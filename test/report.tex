\documentclass[a4paper, xelatex]{bxjsarticle}
\XeTeXlinebreaklocale "ja_JP"
\XeTeXlinebreakskip=0em plus 0.1em minus 0.01em
\XeTeXlinebreakpenalty=0
\usepackage{xltxtra}

% fonts
\usepackage{fontspec}
\setmainfont[Scale=MatchLowercase]{NotoSerifJP-Regular.otf}
\setsansfont[Scale=MatchLowercase]{NotoSansJP-Regular.otf}
\setmonofont[Scale=MatchLowercase]{RobotoMono.ttf}

% bibtex
\usepackage{cite}

% 画像
\usepackage{graphicx}

\begin{document}

  \title{ReXeTeXeR}
  \author{Takahito Sueda}
  \date{2020/11/15}
  \maketitle

  \section{はじめに}
  \XeTeX をDockerだけでいい感じにするやつです。
  Podmanにも対応しました。

  \section{数式}
  普通に{\TeX}で書けます。
  \begin{eqnarray}
    2x^2 + 3x & = & 5
  \end{eqnarray}

  \section{引用}
  bibtexを使えます。引用箇所は↓の感じです。Referencesが最後のとこにあります。
  引用テストDL\cite{lecun2015deep}
  引用テストML\cite{michie1994machine}

  \section{画像}
  graphicx使っていけます。
  \begin{center}
    \includegraphics[width=10cm]{img/logo.png}

    ReXeTeXeRのぶちかっこいいロゴ
  \end{center}

  \section{自動監視}
  なんか書くと、コンテナ内のシェルスクリプトがこのファイルの変更を検知して、texのコンパイルコマンドが走ります。

  % bibtex
  \bibliographystyle{junsrt}
  \bibliography{ref.bib}
\end{document}
