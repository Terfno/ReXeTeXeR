\documentclass[a4paper,xelatex,ja=standard]{bxjsarticle}
\XeTeXlinebreaklocale "ja_JP"
\XeTeXlinebreakskip=0em plus 0.1em minus 0.01em
\XeTeXlinebreakpenalty=0
\usepackage{xltxtra}

% fonts
\usepackage{fontspec}
\setmainfont[Scale=MatchLowercase]{NotoSerifJP-Regular}
\setsansfont[Scale=MatchLowercase]{NotoSansJP-Regular}
\setmonofont[Scale=MatchLowercase]{CourierPrime-Regular}

% bibtex
\usepackage{cite}

% 画像
\usepackage{graphicx}

% custom for 津山高専指示のスタイル
% 動いたパッケージ
\usepackage{mathptmx}
% まだ動かないパッケージ
\usepackage{c_thesis}
\usepackage{listings, jlisting}
\usepackage[normalem]{ulem}
\useunder{\uline}{\ul}{}


% % ソースコードの設定/listings の表示設定
\lstset{
  breaklines = true,
  tabsize = 2,
  frame=trbl, % 枠を上下左右に表示する
  showstringspaces=false, % 文字列中のスペースを" "と表示。def:可視
  numbers=left,
  framexleftmargin=6mm, % 行番号をフレーム内に
  numberstyle=\scriptsize, %行番号のサイズ
  stepnumber=1, % 1行おきに行番号を
  numbersep=1zw, % ソースと行番号の間隔
  language = C % 言語設定
}

% タイトル・表紙設定
\title{ReXeTeXeR}
\author{Takahito Sueda}
\date{2020年7月14日}
\kind{\interim} % 中間報告書の場合 / これはc_thesis.sty
% \kind{\preliminary} % 予備審査論文の場合 / これはc_thesis.sty
% \kind{\thesis} % 最終報告書の場合 / これはc_thesis.sty
\affiliation{情報システム系} % これはc_thesis.sty
\adviser{指導教員の氏名} % これはc_thesis.sty

\begin{document}
  % 表紙
  \maketitle
  \kanjiskip=.25zw plus 3pt minus 3pt % 字間設定
  \xkanjiskip=.25zw plus 3pt minus 3pt % 字間設定

  % Abstract
  \begin{abstract}
    Abstractを書く。英語で。
    この論文の要約を英語のみを用いて、200ワード以上1ページ以内にまとめる。
  \end{abstract}

  % 目次
  \tableofcontents

  \section{はじめに} % \section{研究背景}
  \label{sec1}
  研究の必要性と意義、従来その分野の研究状況、研究の目的とその範囲、オリジナリティを主張する範囲などを記載する。
  現状についての共有→課題→解決方法と目標

  \section{先行研究} % \section{既存研究}
  すでに存在している研究について記載する。

  \section{数式}
  普通に{\TeX}で書けます。
  \begin{eqnarray}
    2x_1 + x_2 & = & 5 \\
    2x_2 & = & 2
  \end{eqnarray}

  \section{引用}
  bibtexを使えます。引用箇所は↓の感じです。Referencesが最後のとこにあります。\\
  引用テストDL\cite{lecun2015deep} \\
  引用テストML\cite{michie1994machine}

  \section{画像}
  graphicx使っていけます。
  \begin{center}
    \includegraphics[width=10cm]{img/logo.png} \\
    ReXeTeXeRのぶちかっこいいロゴ
  \end{center}

  \section{自動監視テスト}
  なんか書くと、コンテナ内のシェルスクリプトがこのファイルの変更を検知して、texのコンパイルコマンドが走ります。

  % bibtex
  \bibliographystyle{junsrt}
  \bibliography{ref.bib}
\end{document}
